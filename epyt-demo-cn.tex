% -*- coding: utf-8 -*-
% !TEX program = xelatex

\documentclass[14pt,notheorems]{beamer}

\usetheme[style=beta]{epyt} % alpha, beta, delta, gamma, zeta

\usepackage[UTF8,noindent]{ctex}

\hypersetup{
  pdfpagemode={FullScreen}
}

\newtheorem{theorem}{定理}
\newtheorem{definition}[theorem]{定义}
\newtheorem{example}[theorem]{例子}

\newtheorem*{theorem*}{定理}
\newtheorem*{definition*}{定义}
\newtheorem*{example*}{例子}

\renewcommand{\proofname}{证明}

\title{Epyt Theme for Beamer}
\author{zohooo@yeah.net}

\begin{document}

\begin{frame}[plain]\transboxout
\titlepage
\end{frame}

\begin{frame}\transboxin
\begin{center}
\tableofcontents[hideallsubsections]
\end{center}
\end{frame}

\epytsetup{style=zeta}

\section{简要介绍}

\begin{frame}{简要介绍}\transdissolve
Epyt 是一个简洁美观的 Beamer 演示文稿主题。它有这些特点:\pause
\begin{itemize}[<+->]
\item 结构简洁,只有包含必需元素的底栏,没有顶栏和侧栏。
\item 内容简洁,列表环境和定理环境都使用了简单的形式。
\item 配色简洁,仅仅用到几种背景色和前景色。
\end{itemize}
\end{frame}

\epytsetup{style=gamma}

\section{列表环境}

\begin{frame}[fragile]{有序列表}\transwipe[direction=270]
无序列表前面已经看到,现在来看看有序列表。一个 Beamer 的主题由下列四部分组成:\pause
\begin{enumerate}[<+->]
\item 外部主题,用 \verb!\usebeameroutertheme! 命令;
\item 内部主题,用 \verb!\usebeamerinnertheme! 命令;
\item 颜色主题,用 \verb!\usebeamercolortheme! 命令;
\item 字体主题,用 \verb!\usebeamerfonttheme! 命令。
\end{enumerate}
\end{frame}

\epytsetup{style=delta}

\section{数学环境}

\begin{frame}{例子证明}\transglitter[direction=90]
\begin{example}
用等价无穷小代换证明下面极限:
\[ \lim_{x\to0}\frac{\sin 3x}{\ln(1-2x)}=-\frac{3}{2} \]
\end{example}\pause
\begin{proof}
因为$\sin 3x \sim 3x$,$\ln(1-2x) \sim -2x$,所以我们有
\[ \lim_{x\to0}\frac{\sin 3x}{\ln(1-2x)}=\lim_{x\to0}\frac{3x}{-2x}=-\frac{3}{2}, \]
即等式成立。
\end{proof}
\end{frame}

\epytsetup{style=alpha}

\section{使用说明}

\begin{frame}[fragile]{使用说明}\transblindsvertical
\begin{itemize}
\item 建议在演示文稿中使用大号的字体,例如:
\begin{verbatim}
\documentclass[14pt]{beamer}
\usebeamertheme{epyt}
\end{verbatim}\pause
\item 如果要使用中文,可以用 \verb ctex 宏包,例如:
\begin{verbatim}
\documentclass[14pt]{beamer}
\usebeamertheme{epyt}
\usepacakge[UTF8,noindent]{ctex}
\end{verbatim}\pause
\item 在载入主题时有几种配色风格可以选择。
\end{itemize}
\end{frame}

\end{document}
