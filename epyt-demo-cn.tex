% -*- coding: utf-8 -*-
% !TEX program = xelatex

\documentclass[14pt,notheorems]{beamer}

\usetheme[style=beta]{epyt} % alpha, beta, delta, gamma, zeta

\usepackage[UTF8,noindent]{ctex}

\usepackage{listings}
\lstset{
  basicstyle=\ttfamily,xleftmargin=0em,
  commentstyle=\color{teal},keywordstyle=\color{acolor4},
  language=[LaTeX]{TeX},morekeywords={usetheme,epytsetup}
}

\newcommand{\mylead}[1]{\textcolor{acolor1}{#1}}
\newcommand{\mybold}[1]{\textcolor{acolor2}{#1}}
\newcommand{\mywarn}[1]{\textcolor{acolor3}{#1}}

\newtheorem{theorem}{定理}
\newtheorem{definition}[theorem]{定义}
\newtheorem{example}[theorem]{例子}

\newtheorem*{theorem*}{定理}
\newtheorem*{definition*}{定义}
\newtheorem*{example*}{例子}

\renewcommand{\proofname}{证明}

\begin{document}

\title{简洁美观的幻灯片主题}
\author{无名小卒}

\begin{frame}[plain]\transboxout
\titlepage
\end{frame}

%\begin{frame}\transboxin
%\begin{center}
%\tableofcontents[hideallsubsections]
%\end{center}
%\end{frame}

\section{黑色文字白色背景}

\begin{frame}{简要介绍}
\mylead{Epyt} 是一个简洁美观的 Beamer 演示主题。它有这些特点:\pause
\begin{itemize}[<+->]
\item 结构简洁,只有包含必需元素的底栏,没有顶栏和侧栏。
\item 内容简洁,列表环境和定理环境可以使用简单形式。
\item 配色简洁,仅仅用到几种背景色和前景色。
\end{itemize}
\end{frame}

\begin{frame}{下载安装}
\mylead{Epyt} 主题已经包含在主要的 TeX 发行版中。
\begin{itemize}
  \item 在 MiKTeX 中作为 \mybold{beamer-theme-epyt} 宏包。
  \item 在 TeXLive 中作为 \mybold{beamertheme-epyt} 宏包。
\end{itemize}
你可以在包管理器中安装它。当然也可以直接从
\href{https://www.ctan.org/pkg/beamertheme-epyt}{CTAN} 中下载并安装。
\end{frame}

\begin{frame}[fragile]{开始使用}
要使用 \mylead{epyt} 主题,可以在导言区中加入下面一行。
\begin{lstlisting}
\usetheme{epyt}
\end{lstlisting}\pause
如果要使用中文,可以添加 \verb ctex 宏包,例如:
\begin{lstlisting}
\usepackage[UTF8,noindent]{ctex}
\end{lstlisting}
\end{frame}

\begin{frame}[fragile]{样式选用}
\mylead{Epyt} 主题中包含了多种样式,可以在载入该主题时选用。
\begin{lstlisting}
\usetheme[style=gamma]{epyt}
\end{lstlisting}
\pause 在文档中间可以用下面命令切换到另一个样式。
\begin{lstlisting}
\epytsetup{style=beta}
\end{lstlisting}
\end{frame}

\begin{frame}{可用的样式}
\mylead{Epyt} 主题中所有可用的样式如下所列。
\begin{description}
  \item[alpha] 白色文字,黑色背景
  \item[beta]  黑色文字,白色背景
  \item[delta] 白色文字,蓝色背景
  \item[gamma] 白色文字,绿色背景
  \item[zeta]  白色文字,红色背景
\end{description}
\pause
其中默认样式为 \mybold{beta}。这也是此文档当前正在使用的样式。
\end{frame}

\begin{frame}{预定义的颜色}
每种样式都预先定义了五种强调颜色。当前样式的强调颜色如下所示。
\begin{flushleft}
\textcolor{acolor1}{acolor1}
\textcolor{acolor2}{acolor2}
\textcolor{acolor3}{acolor3}
\textcolor{acolor4}{acolor4}
\textcolor{acolor5}{acolor5}
\end{flushleft}
\pause 每种样式也预先定义了五种填充颜色,如下所示。
\begin{flushleft}
\colorbox{fcolor1}{fcolor1}
\colorbox{fcolor2}{fcolor2}
\colorbox{fcolor3}{fcolor3}
\colorbox{fcolor4}{fcolor4}
\colorbox{fcolor5}{fcolor5}
\end{flushleft}
\pause 用 tikz 绘图时可以使用这些预先定义的颜色。
\end{frame}

\epytsetup{style=gamma}

\begin{frame}[plain]\transboxout
\titlepage
\end{frame}

\section{白色文字绿色背景}

\begin{frame}{预定义的颜色}
当前样式预先定义了五种强调颜色,如下所示。
\begin{flushleft}
\textcolor{acolor1}{acolor1}
\textcolor{acolor2}{acolor2}
\textcolor{acolor3}{acolor3}
\textcolor{acolor4}{acolor4}
\textcolor{acolor5}{acolor5}
\end{flushleft}
当前样式预先定义了五种填充颜色,如下所示。
\begin{flushleft}
\colorbox{fcolor1}{fcolor1}
\colorbox{fcolor2}{fcolor2}
\colorbox{fcolor3}{fcolor3}
\colorbox{fcolor4}{fcolor4}
\colorbox{fcolor5}{fcolor5}
\end{flushleft}
\end{frame}

\begin{frame}[fragile]{有序列表}
无序列表前面已经看到,现在来看看有序列表。一个 Beamer 的主题由下列四部分组成:\pause
\begin{enumerate}[<+->]
\item 外部主题,用 \verb!\usebeameroutertheme! 命令;
\item 内部主题,用 \verb!\usebeamerinnertheme! 命令;
\item 颜色主题,用 \verb!\usebeamercolortheme! 命令;
\item 字体主题,用 \verb!\usebeamerfonttheme! 命令。
\end{enumerate}
\end{frame}

\epytsetup{style=delta}

\begin{frame}[plain]\transboxout
\titlepage
\end{frame}

\section{白色文字蓝色背景}

\begin{frame}{预定义的颜色}
当前样式预先定义了五种强调颜色,如下所示。
\begin{flushleft}
\textcolor{acolor1}{acolor1}
\textcolor{acolor2}{acolor2}
\textcolor{acolor3}{acolor3}
\textcolor{acolor4}{acolor4}
\textcolor{acolor5}{acolor5}
\end{flushleft}
当前样式预先定义了五种填充颜色,如下所示。
\begin{flushleft}
\colorbox{fcolor1}{fcolor1}
\colorbox{fcolor2}{fcolor2}
\colorbox{fcolor3}{fcolor3}
\colorbox{fcolor4}{fcolor4}
\colorbox{fcolor5}{fcolor5}
\end{flushleft}
\end{frame}

\epytsetup{style=alpha}

\begin{frame}[plain]\transboxout
\titlepage
\end{frame}

\section{白色文字黑色背景}

\begin{frame}{预定义的颜色}
当前样式预先定义了五种强调颜色,如下所示。
\begin{flushleft}
\textcolor{acolor1}{acolor1}
\textcolor{acolor2}{acolor2}
\textcolor{acolor3}{acolor3}
\textcolor{acolor4}{acolor4}
\textcolor{acolor5}{acolor5}
\end{flushleft}
当前样式预先定义了五种填充颜色,如下所示。
\begin{flushleft}
\colorbox{fcolor1}{fcolor1}
\colorbox{fcolor2}{fcolor2}
\colorbox{fcolor3}{fcolor3}
\colorbox{fcolor4}{fcolor4}
\colorbox{fcolor5}{fcolor5}
\end{flushleft}
\end{frame}

\begin{frame}{例子证明}
\begin{example}
证明当$x>0$时不等式$\mathrm{e}^x>1+x$成立。
\end{example}\pause
\begin{proof}
令$f(x)=\mathrm{e}^x-x-1$。则当$x>0$时有
$$f'(x)=\mathrm{e}^x-1>0.$$
因此,当$x>0$时$f(x)>f(0)=0$。得证。
\end{proof}
\end{frame}

\epytsetup{style=zeta}

\begin{frame}[plain]\transboxout
\titlepage
\end{frame}

\section{白色文字红色背景}

\begin{frame}{预定义的颜色}
当前样式预先定义了五种强调颜色,如下所示。
\begin{flushleft}
\textcolor{acolor1}{acolor1}
\textcolor{acolor2}{acolor2}
\textcolor{acolor3}{acolor3}
\textcolor{acolor4}{acolor4}
\textcolor{acolor5}{acolor5}
\end{flushleft}
当前样式预先定义了五种填充颜色,如下所示。
\begin{flushleft}
\colorbox{fcolor1}{fcolor1}
\colorbox{fcolor2}{fcolor2}
\colorbox{fcolor3}{fcolor3}
\colorbox{fcolor4}{fcolor4}
\colorbox{fcolor5}{fcolor5}
\end{flushleft}
\end{frame}

\end{document}
